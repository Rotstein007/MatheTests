%! Author = felix
%! Date = 16.02.24

% Preamble
\documentclass[11pt]{article}

% Packages
\usepackage{amsmath}
\usepackage{amsfonts}

% Document
\begin{document}

Test 1, Gruppe 1

    \textbf{1} Wie ist das Karthesische Produkt von $n$ Mengen $M_1, M_2, M_3, \dots, M_n$ definiert?

    \textbf{2} Wie lauten die Gesethe von $De Morgan$ für Mengen?

    \textbf{3} Wie ist die \textbf{Potenzmenge} einer Menge $M$ definiert?

    \textbf{4} Wie sind (a) \textbf{Maximum} und (b) \textbf{Minimum} einer Menge $M \subset \mathbb{R}$ definiert? (c) Gibt es diese immer? (d) Warum ist es für die Definition relevant, dass $M \subset \mathbb{R}$?

    \textbf{5} Wann sind zwei Abbildungen $f(x)$ und $g(x)$ \textbf{gleich}?

    \textbf{6} Was muss eine Abbildung $f : X \rightarrow Y $ erfüllen, damit sie \textbf{injektiv} ist?

    Aufgabe 7 - 9 nicht erkennbar

    \textbf{10} Sei $f : [0,5] \rightarrow [-1,9], f(x)$ = 2 \cdot x - 1 eine Abbildung, Berechnen Sie (a) das Bild der menge [3, 4] (also $f$([3,4]) und (b) das Urbild einer Menge {-1,0,1} (also $f$^{-1}({-1,0,1})).

Test 2, Gruppe 2

    \textbf{1} (a) Wie ist die \textbf{identische Abbildung} auf einer Menge $X$ definiert? (b) Welche zusätzliche bedingung wird an die Menge $X$ gestellt?

    \textbf{2} Existiert zu jeder Abbildung $f : X \rightarrow Y $ eine Umkehrabbildung? Begründen Sie ihre Antwort!

    \textbf{3} Wie ist die Konjunktion zweier Aussagen $A$ und $B$ definiert?

    \textbf{4} Wann sind zwei Aussageformeln $F_1, F_2$ nach der Definition \textbf{gleichweirtig}?

    \textbf{5} Wie lauten die \textbf{Distributivgesetze} der Aussagenlogik?

    \textbf{6} Auf welcher Aussagenlogischen Äquivalenz beruht das Beweisprinzip der Kontraposition?

    \textbf{7} Sei $f : {-2,-1,0,1,2,3} \rightarrow X, f(x) :=  \begin{cases}
                                                                    -2 \cdot x & \text{, falls $x \leq 0$} \\
                                                                    3 \cdot & \text{, sonst}
    \end{cases}$ eine Abbildung. Bestimmen Sie die Menge $X$ so, dass $f$ bijektiv ist.

    \textbf{8} Seien ( f : \mathbb{Z} \rightarrow \mathbb{N} ), ( f(x) = |x| ) und ( g : \mathbb{R} \rightarrow \mathbb{Z} ), ( g(x) := \lfloor x \rfloor ) ((\lfloor x \rfloor) ist die Floor-Funktion, sie gibt immer die zu der gegebenen Zahl nächst kleinere oder gleiche Zahl zurück).
    (a) Wie können Sie die beiden Funktionen korrekt verketten? (b) Warum geht das so und andersherum nicht?
    Sei ( h ) die korrekt Verkettete Funktion von ( f ) und ( g ). Was liefern dann (c) ( h(\pi) ) und (d) ( h(-5\frac{1}{2}) ).

    \textbf{9} Sei \( f : \mathbb{R} \rightarrow \mathbb{R} \), \( f(x) = 5x + 1 \) eine Funktion. (a) Wie lautet die Umkehrfunktion $f$^{-1} zu $f$. (b) Zeigen Sie, dass Ihre Umkehrfunktion wirklich die Umkehrfunktion ist.

    \textbf{10} Zeigen Sie mittels Wahrheitstabelle, dass die Aussageformeln \(F_1(A,B,C) = A \lor (B \land C)\) und \(F_2(A,B,C) = (A \lor B) \land (A \lor C)\) äquivalent sind.

\[
    \begin{array}{ccc|c|c|c|c|c|c|c}
        A & B & C & B \land C & F_1 & A \lor B & A \lor C & F_2 & F_1 \Leftrightarrow F_2 \\
        \hline
        \text{W} & \text{W} & \text{W} & \text{W} & \text{W} & \text{W} & \text{W} & \text{W} & \text{W} \\
        \text{W} & \text{W} & \text{F} & \text{F} & \text{W} & \text{W} & \text{W} & \text{W} & \text{W} \\
        \text{W} & \text{F} & \text{W} & \text{F} & \text{W} & \text{W} & \text{W} & \text{W} & \text{W} \\
        \text{W} & \text{F} & \text{F} & \text{F} & \text{W} & \text{W} & \text{W} & \text{W} & \text{W} \\
        \text{F} & \text{W} & \text{W} & \text{W} & \text{W} & \text{W} & \text{W} & \text{W} & \text{W} \\
        \text{F} & \text{W} & \text{F} & \text{F} & \text{F} & \text{W} & \text{F} & \text{F} & \text{W} \\
        \text{F} & \text{F} & \text{W} & \text{F} & \text{F} & \text{F} & \text{W} & \text{F} & \text{W} \\
        \text{F} & \text{F} & \text{F} & \text{F} & \text{F} & \text{F} & \text{F} & \text{F} & \text{W} \\
    \end{array}
\]

Test 3, Gruppe 3
    \textbf{1} Wie wird eine binäre Relation ( R_1 \subset M \times M ) auf der Menge ( M ) genannt im Gegensatz zu einer Relation ( R_2 \subset N_1 \times N_2 ) mit ( N_1 \neq N_2 )?

    \textbf{2} Was muss eine Relation \( R \subset X \times Y \) erfüllen, um rechtstotal zu sein?

    \textbf{3} Was muss eine homogene Relation \( R \subset M^2 \) erfüllen, um symmetrisch zu sein?

    \textbf{4} Was muss eine homogene Relation \( R \subset M^2 \) erfüllen, um transitiv zu sein?

    \textbf{5} Nennen Sie die Namen der Eigenschaft einer abelschen Gruppe (G, *).

    \textbf{6} Was ist $S_n$ und wie ist es definiert?

    \textbf{7} Sei $M = {0,1,2,3}$. Bauen Sie eine Relation $R \subset M \times M$, sodass Ihre Relation $reflexiv$ ist.

    \textbf{8} Seien \( M = \{0,-1,-2,-3,-4,-5\} \), \( N = \{0,1,2,3,4,5\} \) Mengen und
    \[ R = \{(0,0),(-1,1),(0,2),(-3,3),(0,4),(-5,5)\} \subset M \times N \]
    eine Relation. Kann \( R \) dann auch der Graph einer Funktion sein? Falls ja, wie lautet dann die Definition der Funktion, und falls nein, warum nicht?

    \textbf{9}

\end{document}