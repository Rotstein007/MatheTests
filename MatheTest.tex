%! Author = felix
%! Date = 16.02.24

% Preamble
\documentclass[11pt]{article}

% Packages
\usepackage{amsmath}
\usepackage{amsfonts}
\usepackage[margin=2cm]{geometry}
\usepackage{xcolor}
\usepackage[ngerman]{babel}
\usepackage[utf8]{inputenc}
\usepackage[T1]{fontenc}
\usepackage{lmodern}

% Document
\begin{document}

    \section{Aufgabensammlung}
    Hier sind alle (für die Minitests) relevanten Aufgaben aus der Aufgabensammlung\newline
    \subsection{Mengen}
    \textbf{Wissen}
    1.1.1 - 1.1.4 \& 1.1.7 - 1.1.10\newline
    \textbf{Rechnen}
    1.2.1 - 1.2.5

    \subsection{Abbildungen}
    \textbf{Wissen}
    2.1.1 - 2.1.13\newline
    \textbf{Rechnen}
    2.2.1 - 2.2.7
    
    \subsection{Aussagenlogik}
    \textbf{Wissen}
    3.1.1 - 3.1.17\newline
    \textbf{Rechnen}
    3.2.1 - 3.2.6

    \subsection{Relationen}
    \textbf{Wissen}
    4.1.1 - 4.1.15\newline
    \textbf{Rechnen}
    4.2.1 - 4.2.10

    \subsection{Algebraische Strukturen}
    \textbf{Wissen}
    5.1.1 - 5.1.9 \& 5.1.11 - 5.1.20\newline
    \textbf{Rechnen}
    5.2.3, 5.2.4

    \subsection{Vollständige Induktion}
    \textbf{Wissen}
    6.1.1 - 6.1.11\newline
    \textbf{Rechnen}
    6.2.2 - 6.2.6 \& (6.2.1)

    \subsection{Rechnen mit Komplexen Zahlen}
    \textcolor{orange}{\textbf{Inhalt fehlt!}}

    \subsection{Mengen in der Komplexen Ebene}
    \textcolor{orange}{\textbf{Inhalt fehlt!}}

    \subsection{Folgen \textcolor{red}{\textbf{Nicht Bestandteil der Minitests!}}}

    \subsection{Reihen \textcolor{red}{\textbf{Nicht Bestandteil der Minitests!}}}

\section{Test1, Gruppe 1}

    \textbf{1} Wie ist das Kartesische Produkt von $n$ Mengen $M_1, M_2, M_3, \dots, M_n$ definiert?\newline
    $M_1 \times M_2 \times M_3 \times \ldots \times M_m = \{ (x_1, x_2, x_3, \ldots, x_m) : x_1 \in M_1, x_2 \in M_2, x_3 \in M_3, \ldots, x_m \in M_m \}$\newline

    \textbf{2} Wie lauten die Gesetze von {\itshape De Morgan} für Mengen?\newline
    $M_1^c \cup M_2^c = (M_1 \cap M_2)^c$\newline
    $M_1^c \cap M_2^c = (M_1 \cup M_2)^c$\newline


    \textbf{3} Wie ist die \textbf{Potenzmenge} einer Menge $M$ definiert?\newline
    $P(M) := 2^M := \{ N : N \subset M \}$\newline

    \textbf{4} Wie sind (a) \textbf{Maximum} und (b) \textbf{Minimum} einer Menge $M \subset \mathbb{R}$ definiert? (c) Gibt es diese immer? (d) Warum ist es für die Definition relevant, dass $M \subset \mathbb{R}$?\newline
    a) $y \geq x$\newline
    b) $y \leq x$\newline
    c) Es muss nicht immer Maximum und Minimum geben, allerdings immer Supremum und Infimum.\newline
    d) \textcolor{red}{Lösung Nicht vorhanden!}\newline

    \textbf{5} Wann sind zwei Abbildungen $f(x)$ und $g(x)$ \textbf{gleich}?\newline
    Zwei Funktionen $f(x)$ und $g(x)$ sind gleich, wenn für jeden Wert von $x$ gilt, dass $f(x) = g(x)$.\newline

    \textbf{6} Was muss eine Abbildung $f : X \rightarrow Y $ erfüllen, damit sie \textbf{injektiv} ist?\newline
    Sie darf jedem Element im Definitionsberech $X$ höchstens ein Element im Zielbereich $Y$ zuordnen.\newline

    \textbf{7} Seien \( M_1 = \{0,1,2\} \) und \( M_2 = \{-1, 0\} \). Berechnen Sie \( M_1 \times M_2 \), das kartesische Produkt.\newline
    $M1 \times M2 = \{(0,-1),(0,0),(1,-1),(1,0),(2,-1),(2,0)\}$ \newline

    \textbf{8} Sei \( M = \{0, 1, \triangle\} \). Berechnen Sie \( P(M) \).\newline
    \( P(M) \) = $\{\{0\}\{1\}\{\triangle\}\{0,1\}\{0,\triangle\}\{1,\triangle\}\{0,1,\triangle\}\{\}\}$\newline

    \textbf{9} Seien \( M_1 = \{a, b, 1\} \), \( M_2 = \{1, 2, c\} \) und \( M = \{0, 1, 2, a, b, c\} \). Berechnen Sie:
    \begin{itemize}
        \item \( M_1 \cup M_2 = {a,b,c,1,2}\)
        \item \( M_1 \setminus M_2 = {a,b}\)
        \item \( M_2 \cap M = {1,2,c}\)
    \end{itemize}\newline

    \textbf{10} Sei $f : [0,5] \rightarrow [-1,9], f(x) = 2 \cdot x - 1$ eine Abbildung, Berechnen Sie (a) das Bild der Menge [3, 4] (also $f$([3,4]) und (b) das Urbild einer Menge {-1,0,1} (also $f^{-1}({-1,0,1})$).\newline\newline
    $f(x) = 2x - 1$ \\
    $f(3) = (2 \cdot 3) - 1 = 5 \quad \quad f(4) = (2 \cdot 4) - 1 = 7$ \\\\
    a) \text{ Bildmenge} = [5,7]\\\\
    b)\\$ f^{-1}(-1) = 2x-1$ \\ $ -1 = 2x - 1$ $|$+1\\$ 0 = 2x |:2$\\$x = 0$\\\\
    $ f^{-1}(0) = 2x-1$ \\ $ 0 = 2x - 1 |+1$\\ $ 1 = 2x |:2$\\$ 0,5 = x$\\\\
    $ f^{-1}(1) = 2x-1$ \\ $ 1 = 2x - 1 |+1$\\ $ 2 = 2x |:2$\\$1 = x$\\\\
    $f^{-1}((-1,0,1)) = (0;0,5;1)$

    \section{Test2, Gruppe 2}

    \textbf{1} (a) Wie ist die \textbf{identische Abbildung} auf einer Menge $X$ definiert? (b) Welche zusätzliche Bedingung wird an die Menge $X$ gestellt?\\
    a) $\text{id}(x) := \text{id}_X(x) := x \quad (x \in X)$\\
    b) Es handelt sich um eine nicht leere Menge. $x(x \in X)$\\

    \textbf{2} Existiert zu jeder Abbildung $f : X \rightarrow Y $ eine Umkehrabbildung? Begründen Sie ihre Antwort!\newline
    Nein, die Funktion muss bijektiv sein, um eine Umkehrabbildung zu bilden. Es könnten sonst Definitionslücken entstehen.\newline

    \textbf{3} Wie ist die Konjunktion zweier Aussagen $A$ und $B$ definiert?\newline
    $A \land B$ \text{ über die Wahrheitstabelle:}
    \[
        \begin{array}{c|c|c}
            A & B & A \land B \\
            \hline
            \text{f} & \text{f} & \text{f} \\
            \text{f} & \text{w} & \text{f} \\
            \text{w} & \text{f} & \text{f} \\
            \text{w} & \text{w} & \text{w} \\
        \end{array}
    \]\newline\newline


    \textbf{4} Wann sind zwei Aussageformeln $F_1, F_2$ nach der Definition \textbf{gleichwertig}?\newline
    Zwei Aussageformeln $F_1$ und $F_2$ heißen gleichwertig, wenn für alle möglichen Besetzungen von Wahrheitswerten in $F_1$ und $F_2$ die Aussage $F_1 \Leftrightarrow F_2$ immer wahr ist.\\

    \textbf{5} Wie lauten die \textbf{Distributivgesetze} der Aussagenlogik?\newline
    $A \land (B \lor C) \iff (A \land B) \lor (A \land C)$\\
    $A \lor (B \land C) \iff (A \lor B) \land (A \lor C)$\\


    \textbf{6} Auf welcher Aussagenlogischen Äquivalenz beruht das Beweisprinzip der Kontraposition?\newline
    Beruht auf der folgenden Äquivalenz:\\
    $(B \Rightarrow C) \Leftrightarrow (\neg B \Leftarrow \neg C)$\\

    \textbf{7} Sei $f : \{-2,-1,0,1,2,3\} \rightarrow X, f(x) :=  \begin{cases}
                                                                    -2 \cdot x & \text{, falls $x \leq 0$} \\
                                                                    3 \cdot x & \text{, sonst}
    \end{cases}$ eine Abbildung. Bestimmen Sie die Menge $X$ so, dass $f$ bijektiv ist.\newline
    $-2 \cdot -2 = 4$\\
    $-2 \cdot -1 = 2$\\
    $-2 \cdot 0 = 0$\\
    $3 \cdot 1 = 3$\\
    $3 \cdot 2 = 6$\\
    $3 \cdot 3 = 9$\\
    $X := \{4,2,0,3,6,9\}$\\

    \textbf{8} Seien \( f : \mathbb{Z} \rightarrow \mathbb{N} \), \( f(x) = |x| \) und \( g : \mathbb{R} \rightarrow \mathbb{Z} \), \( g(x) := \lfloor x \rfloor \) (\(\lfloor x \rfloor\) ist die Floor-Funktion, sie gibt immer die zu der gegebenen Zahl nächst kleinere oder gleiche Zahl zurück).
    (a) Wie können Sie die beiden Funktionen korrekt verketten? (b) Warum geht das so und andersherum nicht?
    Sei ( h ) die korrekt Verkettete Funktion von ( f ) und ( g ). Was liefern dann (c) ( $h(\pi)$ ) und (d) ( $h(-5\frac{1}{2})$ ).\newline
    a) \( f \circ g(x) \)\\
    b) Wenn man die beiden Funktionen anders herum verketten würde, so würde man den \textit{Ausgang} von f in den \textit{Eingang} von g leiten. Und das geht nicht von $\mathbb{N}$ nach $\mathbb{R}$, weil die Anzahl der Elemente in den Mengen unterschiedlich ist, da $\mathbb{N}$ abzählbar ist und $\mathbb{R}$ überabzählbar ist\\
    c) 3\\
    d) 6\\

    \textbf{9} Sei \( f : \mathbb{R} \rightarrow \mathbb{R} \), \( f(x) = 5x + 1 \) eine Funktion. (a) Wie lautet die Umkehrfunktion $f^{-1}$ zu $f$. (b) Zeigen Sie, dass Ihre Umkehrfunktion wirklich die Umkehrfunktion ist.\newline
    a) $f^{-1}(x) = x \cdot (\frac{y-1}{5})$\\
    b) \(f \circ f^{-1}(x) = f^{-1} \circ f(x)\)\\\

    \textbf{10} Zeigen Sie mittels Wahrheitstabelle, dass die Aussageformeln \(F_1(A,B,C) = A \lor (B \land C)\) und \(F_2(A,B,C) = (A \lor B) \land (A \lor C)\) äquivalent sind.

\[
    \begin{array}{ccc|c|c|c|c|c|c|c}
        A & B & C & B \land C & F_1 & A \lor B & A \lor C & F_2 & F_1 \Leftrightarrow F_2 \\
        \hline
        \text{W} & \text{W} & \text{W} & \text{W} & \text{W} & \text{W} & \text{W} & \text{W} & \text{W} \\
        \text{W} & \text{W} & \text{F} & \text{F} & \text{W} & \text{W} & \text{W} & \text{W} & \text{W} \\
        \text{W} & \text{F} & \text{W} & \text{F} & \text{W} & \text{W} & \text{W} & \text{W} & \text{W} \\
        \text{W} & \text{F} & \text{F} & \text{F} & \text{W} & \text{W} & \text{W} & \text{W} & \text{W} \\
        \text{F} & \text{W} & \text{W} & \text{W} & \text{W} & \text{W} & \text{W} & \text{W} & \text{W} \\
        \text{F} & \text{W} & \text{F} & \text{F} & \text{F} & \text{W} & \text{F} & \text{F} & \text{W} \\
        \text{F} & \text{F} & \text{W} & \text{F} & \text{F} & \text{F} & \text{W} & \text{F} & \text{W} \\
        \text{F} & \text{F} & \text{F} & \text{F} & \text{F} & \text{F} & \text{F} & \text{F} & \text{W} \\
    \end{array}
\]

\section{Test3, Gruppe 2}
    \textbf{1} Wie wird eine binäre Relation $( R_1 \subset M \times M )$ auf der Menge ( M ) genannt im Gegensatz zu einer Relation $( R_2 \subset N_1 \times N_2 )$ mit $( N_1 \neq N_2 )$?\newline
    {\itshape Homogene Relation}\newline

    \textbf{2} Was muss eine Relation \( R \subset X \times Y \) erfüllen, um rechtstotal zu sein?\newline
    \forall y \in Y , \exists x \in X : (x, y) \in R\\

    \textbf{3} Was muss eine homogene Relation \( R \subset M^2 \) erfüllen, um symmetrisch zu sein?\newline
    \forall x,y \in M gilt: (x,y) \in R \rightarrow (y,x) \in R \newline

    \textbf{4} Was muss eine homogene Relation \( R \subset M^2 \) erfüllen, um transitiv zu sein?\\
    transitiv, wenn für alle x,y,z \(\in\) M gilt: (x,y) \(\in\) R und (y,z) \(\in\) R \rightarrow (x,z) \(\in\) R\\

    \textbf{5} Nennen Sie die Namen der Eigenschaft einer abelschen Gruppe (G, *).\newline
    Kommutativ (macht Gruppe zur Abelsche Gruppe)\\
    Assoziativ\\
    Neutrales Element\\
    Inverses Element\\

    \textbf{6} Was ist $S_n$ und wie ist es definiert?\newline
    \textcolor{red}{Lösung Nicht vorhanden!}\newline

    \textbf{7} Sei $M = {0,1,2,3}$. Bauen Sie eine Relation $R \subset M \times M$, sodass Ihre Relation {\itshape reflexiv} ist.\newline
\begin{tabular}{c|cccc}
    & 0 & 1 & 2 & 3 \\
    \hline
    0 & 0,0 & & & \\
    1 & & 1,1 & & \\
    2 & & & 2,2 & \\
    3 & & & & 3,3 \\
\end{tabular}\newline
R := {(0,0) (1,1) (2,2) (3,3)}\newline


    \textbf{8} Seien \( M = \{0,-1,-2,-3,-4,-5\} \), \( N = \{0,1,2,3,4,5\} \) Mengen und
    \[ R = \{(0,0),(-1,1),(0,2),(-3,3),(0,4),(-5,5)\} \subset M \times N \]
    eine Relation. Kann \( R \) dann auch der Graph einer Funktion sein? Falls ja, wie lautet dann die Definition der Funktion, und falls nein, warum nicht?\newline
    Nein, allerdings: \textcolor{red}{Lösung Nicht vorhanden!}\newline

    \textbf{9} Gegeben sind die beiden Permutationen \(\sigma_{1} = \begin{pmatrix} 1 & 2 & 3 \\ 3 & 1 & 2 \end{pmatrix}\) und \(\sigma_{2} = \begin{pmatrix} 1 & 2 & 3 \\ 3 & 2 & 1 \end{pmatrix}\) aus der Gruppe \( (S_3, \circ) \). Berechnen Sie (a) \(\sigma_{1} \circ \sigma_{2}\), (b) \(\sigma_{2} \circ \sigma_{1}\) sowie (c) \((\sigma_{1})^{-1}\) und (d) \((\sigma_{2})^{-1}\).
\textbf{9} Gegeben sind die beiden Permutationen \(\sigma_{1} = \begin{pmatrix} 1 & 2 & 3 \\ 3 & 1 & 2 \end{pmatrix}\) und \(\sigma_{2} = \begin{pmatrix} 1 & 2 & 3 \\ 3 & 2 & 1 \end{pmatrix}\) aus der Gruppe \( (S_3, \circ) \). Berechnen Sie
\newline\newline\newline Hier ist glaube ich noch irgendwo ein Fehler!\newline
(a) \(\sigma_{1} \circ \sigma_{2}\):

\[
    \begin{pmatrix}
        1 & 2 & 3 \\
        3 & 1 & 2 \\
    \end{pmatrix}
    \circ
    \begin{pmatrix}
        1 & 2 & 3 \\
        3 & 2 & 1 \\
    \end{pmatrix}
    =
    \begin{pmatrix}
        1 & 2 & 3 \\
        2 & 1 & 3 \\
    \end{pmatrix}
\]

(b) \(\sigma_{2} \circ \sigma_{1}\):

\[
    \begin{pmatrix}
        1 & 2 & 3 \\
        3 & 2 & 1 \\
    \end{pmatrix}
    \circ
    \begin{pmatrix}
        1 & 2 & 3 \\
        3 & 1 & 2 \\
    \end{pmatrix}
    =
    \begin{pmatrix}
        1 & 2 & 3 \\
        2 & 1 & 3 \\
    \end{pmatrix}
\]

(c) \((\sigma_{1})^{-1}\):

\[
    (\sigma_{1})^{-1} = \begin{pmatrix} 1 & 2 & 3 \\ 2 & 3 & 1 \end{pmatrix}
\]

(d) \((\sigma_{2})^{-1}\):

\[
    (\sigma_{2})^{-1} = \begin{pmatrix} 1 & 2 & 3 \\ 3 & 2 & 1 \end{pmatrix}
\]


\textbf{10} Zeigen Sie mittels folgender Wahrheitstabelle, dass die Aussageformeln \(F_1(A,B,C) = A \Rightarrow (B \lor C)\) und \(F_2(A,B,C) = \neg(A \land \neg(B \lor C))\) äquivalent sind.

\[
    \begin{array}{ccc|c|c|c|c|c|c}
        A & B & C & B \lor C & F_1 & \neg(B \lor C) & A \land \neg(B \lor C) & F_2 & F_1 \Leftrightarrow F_2 \\
        \hline
        W & W & W & W & W & F & F & W & W \\
        W & W & F & W & W & F & F & W & W \\
        W & F & W & W & W & F & F & W & W \\
        W & F & F & F & F & W & W & W & W \\
        F & W & W & W & W & F & F & F & W \\
        F & W & F & W & W & F & F & W & W \\
        F & F & W & W & W & F & F & W & W \\
        F & F & F & F & W & W & F & W & W \\
    \end{array}
\]

\section{Test4, Gruppe 2}

    \textbf{1} Sei \((K, \oplus, \odot)\) ein Körper und sei \(k \in K_0 := K \setminus \{0\}\). Wie bezeichnen / schreiben wir allgemein das Inverse von \(k\) bezüglich der Verknüpfung \(\odot\)?\newline
    \textcolor{red}{Lösung Nicht vorhanden!}\newline

    \textbf{2} Sei \((K, \oplus, \odot)\) ein Körper und sei \(k \in K\). Wie bezeichnen oder schreiben wir allgemein das Inverse von \(k\) bezüglich der Verknüpfung \(\oplus\)?\newline
    \textcolor{red}{Lösung Nicht vorhanden!}\newline

    \textbf{3} Unter welcher Bedingung ist der Restklassenring \(\mathbb{Z}_m\) ein Körper?\newline
    \textcolor{red}{Lösung Nicht vorhanden!}\newline

    \textbf{4} (a) Aus welchen drei Abschnitten besteht ein Induktions-Beweis? (b) Skizzieren Sie was in jedem der Abschnitte passiert.\newline\newline
    \textbf{Induktionsanfang} - Einsetzen des kleinsten Wertes für die Variable, zeigen dass Beide Seiten gleich sind. \(n \in M\) : (bewiesene Gleichung)\newline
    \textbf{Induktionsannahme} - Gleichung as der Aufgabenstellung\newline
    \textbf{Induktionsschritt} - $N \rightarrow N+1\newline$
    
    \textbf{5} Wie lautet die \textbf{Gaußsche Summenformel} (Inklusive Verbedingungen)?\newline
    \textcolor{red}{Lösung Nicht vorhanden!}\newline

    \textbf{6} Wie lautet der \textbf{binomische Satz} für \( a, b \in \mathbb{R} \) und \( n \in \mathbb{N}_0 \)?\newline
    \textcolor{red}{Lösung Nicht vorhanden!}\newline

    \textbf{7} Gegeben sei der endliche Restklassenkörper \((\mathbb{Z}_5, +, \cdot)\). Geben Sie die Inversen der Multiplikation (\(\cdot\)) an (soweit diese existieren). (Ergebnisse in Standardrepräsentanten)\newline
    \textcolor{red}{Lösung Nicht vorhanden!}\newline

    \textbf{8} \[
                   \begin{gathered}
                       \text{Berechnen Sie das folgende:} \\
                       \left([4]_1 1\right)^{-1} + \left([5]_1 1\right)^{-1} \cdot [-10]_1 1
                   \end{gathered}
    \]\newline
    \textcolor{red}{Lösung Nicht vorhanden!}\newline

    \textbf{9} Berechnen Sie \(\binom{6}{4}\):

    \[
        \binom{6}{4} = \frac{6!}{4!(6-4)!} = \frac{6 \times 5 \times 4 \times 3 \times 2 \times 1}{4 \times 3 \times 2 \times 1 \times 2 \times 1} = \frac{720}{24 \cdot 2} = 15
    \]\newline


    \textbf{10} (c) Wie viele Mögliche sechsstellige Matrikelnummern gibt es im Dezimalsystem (Bestehend aus den Ziffern {0,1,2,3,4,5,6,7,8,9})? (Ziehen (a) mit/ohne Zurücklegen, (b) mit/ohne Beachtung der Reihenfolge?)\newline
    $n = 10$\newline
    $k = 6$
    $n^{k}$ = $10^{6}$ = $1.000.000$

\section{Test5, Gruppe 2}

    \textbf{1} Wie ist die Menge der komplexen Zahlen (\(\mathbb{C}\)) mittels der reellen Zahlen (\(\mathbb{R}\)) definiert?\newline
    $C := \{ a + ib : a \in \mathbb{R}, b \in \mathbb{R} \}$\newline

    \textbf{2} Sei \( z = a + ib \in \mathbb{C} \), \( a,b \in \mathbb{R} \) eine komplexe Zahl. Wie sind \( \text{Re}(z) \) und \( \text{Im}(z) \) definiert?\newline
    $Re(z) := a$\newline
    $Im(z) := b$\newline

    \textbf{3} Wie ist die \textbf{Multiplikation} zweier komplexer Zahlen \( z = a + ib, w = c + id \in \mathbb{C} \), \( a,b,c,d \in \mathbb{R} \) definiert? \( z \cdot w = \ldots \)\newline
    \textcolor{red}{Lösung Nicht vorhanden!}\newline

    \textbf{4} Wie ist der \textbf{Betrag} einer komplexen Zahl \( z = a + ib \in \mathbb{C} \), \( a,b \in \mathbb{R} \) definiert?\newline
    $\rvert z\rvert := \sqrt{a^2 + b^2}$\newline

    \textbf{5} Welchen Grad hat das folgende Polynom? \(f(z) = (z^2 - z + \frac{1}{4})(z - (1 - i))^2\)\newline
    \textcolor{red}{Lösung Nicht vorhanden!}\newline

    \textbf{6} Betrachten Sie das folgende Polynom \(f(z) = z^7 - 15z^6 + 97z^5 - 375z^4 + 1103z^3 - 2305z^2 + 2799z - 1305\) mit den Nullstellen \(n_0 = 1\), \(n_1 = i3\), \(n_3 = 2 - i\), \(n_5 = 5 + i2\). Geben Sie die fehlenden Nullstellen \(n_2\), \(n_4\) und \(n_6\) an.\newline
    \textcolor{red}{Lösung Nicht vorhanden!}\newline

    \textbf{7} Vervollständigen Sie:
    (Fundamentalsatz der \underline{\hspace{2cm}} und \underline{\hspace{2cm}} satz) Es sei
    \( P(z) = \sum_{v=0}^{d} a_v z \)
    ein Polynom vom \( \underline{\hspace{0.5cm}} > 0 \). Dann existieren \( z_1, ..., z_k \in \underline{\hspace{0.5cm}} \) und zugehörige \( \alpha_1, ..., \alpha_k \in \underline{\hspace{0.5cm}} \), so dass
    \( P(z) = a_d (z-\underline{\hspace{0.5cm}})^{\underline{\hspace{0.5cm}}} (z-\underline{\hspace{0.5cm}})^{\underline{\hspace{0.5cm}}} ... (z-\underline{\hspace{0.5cm}})^{\underline{\hspace{0.5cm}}} \)
    gilt, dabei ist \( \alpha_1 + \alpha_2 + ... + \alpha_k = \underline{\hspace{0.5cm}} \).\newline

    \textbf{8} Wie lautet die allgemeine Formel für eine Kreisscheibe mir Radius $r$ und Mittelpunkt $a + ib$ in der Komplexen-Ebene?\newline
    \textcolor{red}{Lösung Nicht vorhanden!}\newline

    \textbf{9} Zeichnen Sie die folgende Menge in der komplexen Ebene:
    \[ M := \{ z \in \mathbb{C} : \text{Re}(z) \geq -\frac{5}{2} \land \text{Im}(z) < \frac{\pi}{2} \} \]\newline
    Lösung in Arbeit\newline

    \textbf{10} Beschreiben Sie die im Folgenden skizzierte Menge mittels der Mengenschreibweise.\newline
    Lösung in Arbeit\newline

\end{document}