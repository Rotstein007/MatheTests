%! Author = felix
%! Date = 16.02.24

% Preamble
\documentclass[11pt]{article}

% Packages
\usepackage{amsmath}
\usepackage{amsfonts}
\usepackage[margin=2cm]{geometry}

% Document
\begin{document}

Test 1, Gruppe 1

    \textbf{1} Wie ist das Karthesische Produkt von $n$ Mengen $M_1, M_2, M_3, \dots, M_n$ definiert?

    \textbf{2} Wie lauten die Gesethe von $De Morgan$ für Mengen?

    \textbf{3} Wie ist die \textbf{Potenzmenge} einer Menge $M$ definiert?

    \textbf{4} Wie sind (a) \textbf{Maximum} und (b) \textbf{Minimum} einer Menge $M \subset \mathbb{R}$ definiert? (c) Gibt es diese immer? (d) Warum ist es für die Definition relevant, dass $M \subset \mathbb{R}$?

    \textbf{5} Wann sind zwei Abbildungen $f(x)$ und $g(x)$ \textbf{gleich}?

    \textbf{6} Was muss eine Abbildung $f : X \rightarrow Y $ erfüllen, damit sie \textbf{injektiv} ist?

    \textbf{7} Seien \( M_1 = \{0,1,2\} \) und \( M_2 = \{-1, 0\} \). Berechnen Sie \( M_1 \times M_2 \), das kartesische Produkt.

    \textbf{8} Sei \( M = \{0, 1, \triangle\} \). Berechnen Sie \( P(M) \).

    \textbf{9} Seien \( M_1 = \{a, b, 1\} \), \( M_2 = \{1, 2, c\} \) und \( M = \{0, 1, 2, a, b, c\} \). Berechnen Sie:
    \begin{itemize}
        \item \( M_1 \cup M_2 =\)
        \item \( M_1 \setminus M_2 =\)
        \item \( M_2 \cap M =\)
    \end{itemize}

\textbf{10} Sei $f : [0,5] \rightarrow [-1,9], f(x) = 2 \cdot x - 1$ eine Abbildung, Berechnen Sie (a) das Bild der menge [3, 4] (also $f$([3,4]) und (b) das Urbild einer Menge {-1,0,1} (also $f$^{-1}({-1,0,1})).

Test 2, Gruppe 2

    \textbf{1} (a) Wie ist die \textbf{identische Abbildung} auf einer Menge $X$ definiert? (b) Welche zusätzliche bedingung wird an die Menge $X$ gestellt?

    \textbf{2} Existiert zu jeder Abbildung $f : X \rightarrow Y $ eine Umkehrabbildung? Begründen Sie ihre Antwort!

    \textbf{3} Wie ist die Konjunktion zweier Aussagen $A$ und $B$ definiert?

    \textbf{4} Wann sind zwei Aussageformeln $F_1, F_2$ nach der Definition \textbf{gleichweirtig}?

    \textbf{5} Wie lauten die \textbf{Distributivgesetze} der Aussagenlogik?

    \textbf{6} Auf welcher Aussagenlogischen Äquivalenz beruht das Beweisprinzip der Kontraposition?

    \textbf{7} Sei $f : {-2,-1,0,1,2,3} \rightarrow X, f(x) :=  \begin{cases}
                                                                    -2 \cdot x & \text{, falls $x \leq 0$} \\
                                                                    3 \cdot & \text{, sonst}
    \end{cases}$ eine Abbildung. Bestimmen Sie die Menge $X$ so, dass $f$ bijektiv ist.

    \textbf{8} Seien \( f : \mathbb{Z} \rightarrow \mathbb{N} \), \( f(x) = |x| \) und \( g : \mathbb{R} \rightarrow \mathbb{Z} \), \( g(x) := \lfloor x \rfloor \) (\(\lfloor x \rfloor\) ist die Floor-Funktion, sie gibt immer die zu der gegebenen Zahl nächst kleinere oder gleiche Zahl zurück).
    (a) Wie können Sie die beiden Funktionen korrekt verketten? (b) Warum geht das so und andersherum nicht?
    Sei ( h ) die korrekt Verkettete Funktion von ( f ) und ( g ). Was liefern dann (c) ( h(\pi) ) und (d) ( h(-5\frac{1}{2}) ).

    \textbf{9} Sei \( f : \mathbb{R} \rightarrow \mathbb{R} \), \( f(x) = 5x + 1 \) eine Funktion. (a) Wie lautet die Umkehrfunktion $f^{-1}$ zu $f$. (b) Zeigen Sie, dass Ihre Umkehrfunktion wirklich die Umkehrfunktion ist.

    \textbf{10} Zeigen Sie mittels Wahrheitstabelle, dass die Aussageformeln \(F_1(A,B,C) = A \lor (B \land C)\) und \(F_2(A,B,C) = (A \lor B) \land (A \lor C)\) äquivalent sind.

\[
    \begin{array}{ccc|c|c|c|c|c|c|c}
        A & B & C & B \land C & F_1 & A \lor B & A \lor C & F_2 & F_1 \Leftrightarrow F_2 \\
        \hline
        \text{W} & \text{W} & \text{W} & \text{W} & \text{W} & \text{W} & \text{W} & \text{W} & \text{W} \\
        \text{W} & \text{W} & \text{F} & \text{F} & \text{W} & \text{W} & \text{W} & \text{W} & \text{W} \\
        \text{W} & \text{F} & \text{W} & \text{F} & \text{W} & \text{W} & \text{W} & \text{W} & \text{W} \\
        \text{W} & \text{F} & \text{F} & \text{F} & \text{W} & \text{W} & \text{W} & \text{W} & \text{W} \\
        \text{F} & \text{W} & \text{W} & \text{W} & \text{W} & \text{W} & \text{W} & \text{W} & \text{W} \\
        \text{F} & \text{W} & \text{F} & \text{F} & \text{F} & \text{W} & \text{F} & \text{F} & \text{W} \\
        \text{F} & \text{F} & \text{W} & \text{F} & \text{F} & \text{F} & \text{W} & \text{F} & \text{W} \\
        \text{F} & \text{F} & \text{F} & \text{F} & \text{F} & \text{F} & \text{F} & \text{F} & \text{W} \\
    \end{array}
\]

Test 3, Gruppe 3
    \textbf{1} Wie wird eine binäre Relation $( R_1 \subset M \times M )$ auf der Menge ( M ) genannt im Gegensatz zu einer Relation ( R_2 \subset N_1 \times N_2 ) mit ( N_1 \neq N_2 )?

    \textbf{2} Was muss eine Relation \( R \subset X \times Y \) erfüllen, um rechtstotal zu sein?

    \textbf{3} Was muss eine homogene Relation \( R \subset M^2 \) erfüllen, um symmetrisch zu sein?

    \textbf{4} Was muss eine homogene Relation \( R \subset M^2 \) erfüllen, um transitiv zu sein?

    \textbf{5} Nennen Sie die Namen der Eigenschaft einer abelschen Gruppe (G, *).

    \textbf{6} Was ist $S_n$ und wie ist es definiert?

    \textbf{7} Sei $M = {0,1,2,3}$. Bauen Sie eine Relation $R \subset M \times M$, sodass Ihre Relation $reflexiv$ ist.

    \textbf{8} Seien \( M = \{0,-1,-2,-3,-4,-5\} \), \( N = \{0,1,2,3,4,5\} \) Mengen und
    \[ R = \{(0,0),(-1,1),(0,2),(-3,3),(0,4),(-5,5)\} \subset M \times N \]
    eine Relation. Kann \( R \) dann auch der Graph einer Funktion sein? Falls ja, wie lautet dann die Definition der Funktion, und falls nein, warum nicht?

    \textbf{9} Gegeben sind die beiden Permutationen \(\sigma_{1} = \begin{pmatrix} 1 & 2 & 3 \\ 3 & 1 & 2 \end{pmatrix}\) und \(\sigma_{2} = \begin{pmatrix} 1 & 2 & 3 \\ 3 & 2 & 1 \end{pmatrix}\) aus der Gruppe \( (S_3, \circ) \). Berechnen Sie (a) \(\sigma_{1} \circ \sigma_{2}\), (b) \(\sigma_{2} \circ \sigma_{1}\) sowie (c) \((\sigma_{1})^{-1}\) und (d) \((\sigma_{2})^{-1}\).

    \textbf{10} Zeigen Sie mittels folgender Wahrheitstabelle, dass die Aussageformeln \(F_1(A,B,C) = A \Rightarrow (B \lor C)\) und \(F_2(A,B,C) = \neg(A \land \neg(B \lor C))\) äquivalent sind.

\[
    \begin{array}{ccc|c|c|c|c|c|c}
        A & B & C & B \lor C & F_1 & \neg(B \lor C) & A \land \neg(B \lor C) & F_2 & F_1 \Leftrightarrow F_2 \\
        \hline
        W & W & W & W & W & F & F & W & W \\
        W & W & F & W & W & F & F & W & W \\
        W & F & W & W & W & F & F & W & W \\
        W & F & F & F & F & W & W & W & W \\
        F & W & W & W & W & F & F & F & W \\
        F & W & F & W & W & F & F & W & W \\
        F & F & W & W & W & F & F & W & W \\
        F & F & F & F & W & W & F & W & W \\
    \end{array}
\]

Test 4, Gruppe 2

    \textbf{1} Sei \((K, \oplus, \odot)\) ein Körper und sei \(k \in K_0 := K \setminus \{0\}\). Wie bezeichnen / schreiben wir allgemein das Inverse von \(k\) bezüglich der Verknüpfung \(\odot\)?

    \textbf{2} Sei \((K, \oplus, \odot)\) ein Körper und sei \(k \in K\). Wie bezeichnen oder schreiben wir allgemein das Inverse von \(k\) bezüglich der Verknüpfung \(\oplus\)?

    \textbf{3} Unter welcher Bedingung ist der Restklassenring \(\mathbb{Z}_m\) ein Körper?

    \textbf{4} (a) Aus welchen drei Abschnitten besteht ein Induktions-Beweis? (b) Skizzieren Sie was in jedem der Abschnitte passiert.
    
    \textbf{5} Wie lautet die \textbf{Gaußsche Summenformel} (Inklusive Verbedingungen)?

    \textbf{6} Wie lautet der \textbf{binomische Satz} für \( a, b \in \mathbb{R} \) und \( n \in \mathbb{N}_0 \)?

    \textbf{7} Gegeben sei der endliche Restklassenkörper \((\mathbb{Z}_5, +, \cdot)\). Geben Sie die Inversen der Multiplikation (\(\cdot\)) an (soweit diese existieren). (Ergebnisse in Standardrepräsentanten)

    \textbf{8} \[
                   \begin{gathered}
                       \text{Berechnen Sie das folgende:} \\
                       \left([4]_1 1\right)^{-1} + \left([5]_1 1\right)^{-1} \cdot [-10]_1 1
                   \end{gathered}
\]

    \textbf{9} \[
                   \begin{gathered}
                       \text{Berechnen Sie } \binom{6}{4}
                   \end{gathered}
\]

    \textbf{10} (c) Wie viele Mögliche schsstellige Metrikelnummern gibt es im Dezimalsystem (Bestehend aus den Ziffern {0,1,2,3,4,5,6,7,8,9})? (Ziehen (a) mit/ohne Zurücklegen, (b) mit/ohne Beachtung der Reihenfolge?)

Test 5, Gruppe 2

    \textbf{1} Wie ist die Menge der komplexen Zahlen (\(\mathbb{C}\)) mittels der reellen Zahlen (\(\mathbb{R}\)) definiert?

    \textbf{2} Sei \( z = a + ib \in \mathbb{C} \), \( a,b \in \mathbb{R} \) eine komplexe Zahl. Wie sind \( \text{Re}(z) \) und \( \text{Im}(z) \) definiert?

    \textbf{3} Wie ist die \textbf{Multiplikation} zweier komplexer Zahlen \( z = a + ib, w = c + id \in \mathbb{C} \), \( a,b,c,d \in \mathbb{R} \) definiert? \( z \cdot w = \ldots \)

    \textbf{4} Wie ist der \textbf{Betrag} einer komplexen Zahl \( z = a + ib \in \mathbb{C} \), \( a,b \in \mathbb{R} \) definiert?

    \textbf{5} Welchen Grad hat das folgende Polynom? \(f(z) = (z^2 - z + \frac{1}{4})(z - (1 - i))^2\)

    \textbf{6} Betrachten Sie das folgende Polynom \(f(z) = z^7 - 15z^6 + 97z^5 - 375z^4 + 1103z^3 - 2305z^2 + 2799z - 1305\) mit den Nullstellen \(n_0 = 1\), \(n_1 = i3\), \(n_3 = 2 - i\), \(n_5 = 5 + i2\). Geben Sie die fehlenden Nullstellen \(n_2\), \(n_4\) und \(n_6\) an.

    \textbf{7} Vervollständigen Sie:
(Fundamentalsatz der \_\_\_\_\_\_\_\_\_\_\_\_ und \_\_\_\_\_\_\_\_\_\_\_\_ satz) Es sei
\( P(z) = \sum_{v=0}^{d} a_v z \)
ein Polynom vom \( \_\_\_\_ > 0 \). Dann existieren \( z_1, ..., z_k \in \_\_ \) und zugehörige \( \alpha_1, ..., \alpha_k \in \_\_ \), so dass
\( P(z) = a_d (z-\_\_)^{\_\_} (z-\_\_)^{\_\_} ... (z-\_\_)^{\_\_} \)
gilt, dabei ist \( \alpha_1 + \alpha_2 + ... + \alpha_k = \_\_ \).

    \textbf{8} Wie lautet die allgemeine Formel für eine Kreisscheibe mir Radius $r$ und Mittelpunkt $a + ib$ in der Komplexen-Ebene?

    \textbf{9} Zeichnen Sie die folgende Menge in der komplexen Ebene:
\[ M := \{ z \in \mathbb{C} : \text{Re}(z) \geq -\frac{5}{2} \land \text{Im}(z) < \frac{\pi}{2} \} \]

    \textbf{10} Beschreiben Sie die im folgenden Skizzierte Menge mittels der Mengenschreibweise.

\end{document}